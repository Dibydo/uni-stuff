\documentclass [12pt]{article}


\usepackage{ucs}
\usepackage[utf8x]{inputenc} %Поддержка UTF8
\usepackage{cmap} % Улучшенный поиск русских слов в полученном pdf-файле
\usepackage[english,russian]{babel} %Пакет для поддержки русского и английского языка
\usepackage{graphicx} %Поддержка графиков
\usepackage{float} %Поддержка float-графиков
\usepackage[left=20mm,right=15mm, top=20mm,bottom=20mm,bindingoffset=0cm]{geometry}
\usepackage{mathtools} 
\DeclarePairedDelimiter{\abs}{\lvert}{\rvert}
\renewcommand{\baselinestretch}{1.2}
 
\usepackage{color} 
\definecolor{deepblue}{rgb}{0,0,0.5}
\definecolor{deepred}{rgb}{0.6,0,0}
\definecolor{deepgreen}{rgb}{0,0.5,0}
\definecolor{gray}{rgb}{0.5,0.5,0.5}

\DeclareFixedFont{\ttb}{T1}{txtt}{bx}{n}{12} % for bold
\DeclareFixedFont{\ttm}{T1}{txtt}{m}{n}{12}  % for normal

\usepackage{listings}
 
\lstset{
	language=Python,
	basicstyle=\ttm,
	otherkeywords={self},             % Add keywords here
	keywordstyle=\ttb\color{deepblue},
	emph={MyClass,__init__},          % Custom highlighting
	emphstyle=\ttb\color{deepred},    % Custom highlighting style
	stringstyle=\color{deepgreen},
	frame=tb,                         % Any extra options here
	showstringspaces=false            % 
}
 
\usepackage{hyperref}
 
\hypersetup{
    bookmarks=true,         % show bookmarks bar?
    unicode=false,          % non-Latin characters in Acrobat’s bookmarks
    pdftoolbar=true,        % show Acrobat’s toolbar?
    pdfmenubar=true,        % show Acrobat’s menu?
    pdffitwindow=false,     % window fit to page when opened
    pdfstartview={FitH},    % fits the width of the page to the window
    pdftitle={My title},    % title
    pdfauthor={Author},     % author
    pdfsubject={Subject},   % subject of the document
    pdfcreator={Creator},   % creator of the document
    pdfproducer={Producer}, % producer of the document
    pdfkeywords={keyword1} {key2} {key3}, % list of keywords
    pdfnewwindow=true,      % links in new PDF window
    colorlinks=true,       % false: boxed links; true: colored links
    linkcolor=black,          % color of internal links (change box color with linkbordercolor)
    citecolor=green,        % color of links to bibliography
    filecolor=magenta,      % color of file links
    urlcolor=cyan           % color of external links
}

\title{}
\date{}
\author{}

\begin{document}
\begin{titlepage}
\thispagestyle{empty}
\begin{center}
Федеральное государственное бюджетное образовательное учреждение высшего профессионального образования \\Московский государственный технический университет имени Н.Э. Баумана

\end{center}
\vfill
\centerline{\large{Лабораторная работа №5}}
\centerline{\large{по курсу <<Численные методы>>}}
\centerline{\large{<<Решение дифференциальных уравнений. }}
\centerline{\large{Метод Рунге-Кутты 4 порядка>>}}
\vfill
\hfill\parbox{5cm} {
           Выполнил:\\
           студент группы ИУ9-62 \hfill \\
           Иванов Георгий\hfill \medskip\\
           Проверила:\\
           Домрачева А.Б.\hfill
       }
\centerline{Москва, 2017}
\clearpage
\end{titlepage}

\textsc{\textbf{Цель:}} 

Анализ метода Рунге-Кутты 4 порядка для решения дифференциальных уравнений, обобщение метода для решения дифференциальных уравнений высших порядков (2 и выше). 

Решение задачи Коши (частного решения) для дифференциального уравнения 2 порядка с помощью этого метода.

\textsc{\textbf{Постановка задачи:}}

\textbf{Дано:}  

1) Неоднородное дифференциальное уравнение 2 порядка с постоянными коэффициентами: $$y''(x)+p(x)y'(x)+q(x)y(x)=f(x)$$ где $p(x)$ и $q(x)$ - постоянные функции, а $f(x)$ - функция, непрерывна на интервале интегрирования $[a;b]$.

2) Начальные условия задачи Коши: $$ y(a)=A, \quad y'(a)=B, \quad x \in [a;b] $$

 \textbf{Найти:} Численное решение данного неоднородного линейного дифференциального уравнения 2 порядка при начальных условиях задачи Коши методом Рунге-Кутты.

\textbf{Тестовый пример:} 

В нашем случае, выберем следующие значения:  $$y(x) = e^x, \quad p(x)=1, \quad q(x)=-1$$

Дифференциальное уравнение примет вид:  $$y''(x)+y'(x)-y(x)=e^x$$

Определим начальные условия задачи Коши: $$ y(0)=e^0=1, \quad y'(0)=e^0=1, \quad x \in [0;1] $$

\textsc{\textbf{Теоретические сведения:}}

Методом Рунге-Кутты четвертого порядка точности называют одношаговый метод, относящийся к широкому классу методов Рунге-Кутты решения задачи Коши для обыкновенных дифференциальных уравнений и их систем.

Этот метод используют для точного расчета стандартных моделей достаточно часто, так как при небольшом объеме вычислений он обладает точностью метода $O^4(h)$, в отличие от стандартного метода Эйлера. Метод является явным, так как $y_{n+1}$ находится по ранее найденным значениям. В зависимости от степени аппроксимации решение одной и той же задачи позволяет получать более точное решение при более крупном шаге, и  приводит к снижению требуемых ресурсов ЭВМ при меньшем количестве шагов. Широко распространён и реализуем классический метод Рунге-Кутты 4 порядка. Можно также повысить порядок вычисления, но это грозит к большим вычислительным трудностям. 

Стоит отметить, что данный метод применим для дифференциальных уравнений 1 порядка, поэтому необходимо обобщить данный метод для уравнений высших порядков при помощи понижения порядка уравнения и переходом к системе дифференциальных уравнений 1 порядка.

Это можно сделать посредством замены переменных $y'=z$. Тогда получим систему:
\begin{equation*}
\begin{cases}
g(x,y,z) = y' = z \\
f(x,y,z) = z' = e^x + y'(x)-y(x)
\end{cases}
\end{equation*}
с начальными условиями $ y(a)=A,  z(a)=B $ для первого и второго уравнения системы.

\textbf{Описание алгоритма:}

Приближенное значение, используемое в методе Рунге-Кутты 4 порядка, в следующих точках вычисляется по итерационной формуле:
$$y_{n+1}=y_{n}+\frac{h}{6}(k_{1}+2k_{2}+2k_{3}+k_{4})	$$ где $n=0,...,N$ - число разбиений отрезка, $y_{n}$ - старое значение, $y_{n+1}$ - новое значение, а коэффициенты $k_{1},k_{2},k_{3},k_{4}$ вычисляются следующим способом:

$$k_1=f(x_n,y_n)$$
$$k_2=f(x_n+\frac{h}{2},y_n+\frac{h}{2}k_1)$$
$$k_3=f(x_n+\frac{h}{2},y_n+\frac{h}{2}k_2)$$
$$k_4=f(x_n+h,y_n+hk_3)$$
где $f(x_n,y_n)$ – значение дифференциального уравнения 1 порядка в точке $x_n$, $h=\frac{b-a}{N}$ -длина шага.


\textsc{\textbf{Практическая реализация:}}

Листинг 1. Метод Рунге-Кутты 4 порядка для решения диф.уравнения
\begin{lstlisting}[language=python]

#!python
# -*- coding: utf-8 -*-

import abc
import math

class RungeKutta(object):
    __metaclass__ = abc.ABCMeta

    def __init__(self,t0,y0):
        self.t = t0
        self.y = y0
        self.yy = [0,0]
        self.y1 = [0,0]
        self.y2 = [0,0]
        self.y3 = [0,0]
        self.y4 = [0,0]

    @abc.abstractclassmethod
    def f(self,t,y):
        pass

    def nextStep(self,dt):
        if dt < 0:
            return None

        self.y1 = self.f(self.t, self.y)
        for i in range(2):
            self.yy[i] = self.y[i] + self.y1[i] * (dt / 2.0)

        self.y2 = self.f(self.t + dt / 2.0, self.yy)
        for i in range(2):
            self.yy[i] = self.y[i] + self.y2[i] * (dt / 2.0)

        self.y3 = self.f(self.t + dt / 2.0, self.yy)
        for i in range(2):
            self.yy[i] = self.y[i] + self.y3[i] * dt

        self.y4 = self.f(self.t + dt, self.yy)
        for i in range(2):
            self.y[i] = self.y[i] + dt / 6.0 *\ 
(self.y1[i] + 2.0 * self.y2[i] + 2.0 * self.y3[i] + self.y4[i])

        self.t += dt

class RungeKuttaImpl(RungeKutta):
    def f(self,t,y):
        fy = [0,0]
        fy[0] = y[1]
        fy[1] = math.exp(t) + y[0] - y[1]
        return fy

def func(x):
    return math.exp(x)

if __name__ == "__main__":
    eq = RungeKuttaImpl(0,[1,1])
    dt = 0.1
    while eq.t <= 1.00:
        print("Solution y'(%.2f)=" %(eq.t) + str(eq.y[1]) + 
		       " y(%.2f)=" %(eq.t) + str(eq.y[0]))
        eq.nextStep(dt)

\end{lstlisting}

Класс RungeKutta является абстрактным,а значит для использования его для решения системы дифференциальных уравнений 1 порядка необходимо переопределить правые части системы. RungeKutta.y - массив решений, где $y[0]$ - само решение, $y[1]$ - первая производная решения. RungeKutta.t - значение $x_n$ на текущем шаге. Чтобы задать начальные условия задачи Коши, необходимо создать объект класса RungeKuttaImpl, причём вторым параметром конструктора является массив начальных значений условий задачи Коши. RungeKutta.nextStep(dt) - вычисление следующего значения решения системы дифференциального уравнения при $x_{i+1}=x_{i}+dt$

\textsc{\textbf{Результаты:}}

Для тестирования полученной программы было выбрано уравнение: $$y''(x)+y'(x)-y(x)=e^x$$
c начальными условиями задачи Коши: $$ y(0)=e^0=1, \quad y'(0)=e^0=1, \quad x \in [0;1] $$

Чтобы проверить точность метода, вычислим погрешность: $$\varepsilon=y^{*}_{n}-y_{n}$$ где $y^{*}_{n}$ - численное решение данного уравнения, $y_{n}$ - полученное решение посредством метода Рунге-Кутты 4 порядка. 

Ниже приведена таблица результата полученной программы для вычисления погрешности (Листинг 1) на указанном методе:

\begin{center}
\begin{tabular}{ |c|c|c|c| }
  \hline
  Значение $x_{n}$ & Численное решение $y^{*}_{n}$ & Полученное решение $y_{n}$ & Погрешность $\varepsilon$ \\ \hline
  0 & 1 & 1.0 & 0.0 \\ \hline
  0.1 & 1.1051707977298888 & 1.1051709180756477 & 1.2034575891384236e-07 \\ \hline
  0.2 & 1.2214025180676706 & 1.2214027581601699 & 2.4009249921519427e-07 \\ \hline
  0.3 & 1.349858446117391 & 1.3498588075760032 & 3.6145861215253206e-07 \\ \hline
  0.4 & 1.4918242110670952 & 1.4918246976412703 & 4.865741751736152e-07 \\ \hline
  0.5 & 1.6487206531791918 & 1.6487212707001282 & 6.175209363856737e-07 \\ \hline
  0.6 & 1.8221180440212896 & 1.8221188003905089 & 7.563692192569249e-07 \\ \hline
  0.7 & 2.0137518022579552 & 2.0137527074704766 & 9.052125213848683e-07 \\ \hline
  0.8 & 2.2255398622919795 & 2.2255409284924674 & 1.0662004878980724e-06 \\ \hline
  0.9 & 2.459601869586057 & 2.4596031111569494 & 1.2415708923185775e-06 \\ \hline
  1.0 & 2.7182803947778686 & 2.718281828459045 & 1.433681176443713e-06 \\ \hline
\end{tabular}
\end{center}

А также приведена таблица результата полученной программы для вычисления значений системы дифференциальных уравнений с начальными условиями задачи Коши (Листинг 1) на указанном методе:

\begin{center}
\begin{tabular}{ |c|c|c| }
  \hline
  Значение $x_{n}$ & Решение $y_{n}$ & Решение $y'_{n}$ \\ \hline
  0 & 1 & 1 \\ \hline
  0.1 & 1.1051707977298888 & 1.1051710632544716 \\ \hline
  0.2 & 1.2214025180676706 & 1.221403039179291 \\ \hline
  0.3 & 1.349858446117391 & 1.3498592176047473 \\ \hline
  0.4 & 1.4918242110670952 & 1.4918252320873335 \\ \hline
  0.5 & 1.6487206531791918 & 1.6487219269906563 \\ \hline
  0.6 & 1.8221180440212896 & 1.8221195777972221 \\ \hline
  0.7 & 2.0137518022579552 & 2.0137536069733986 \\ \hline
  0.8 & 2.2255398622919795 & 2.225541952677753 \\ \hline
  0.9 & 2.459601869586057 & 2.4596042641450997 \\ \hline
  1.0 & 2.7182803947778686 & 2.7182831158605034 \\ \hline
\end{tabular}
\end{center}

\textsc{\textbf{Выводы:}}

В ходе выполнения лабораторной работы был рассмотрен метод решения дифференциальных уравнений, а именно классический метод Рунге-Кутты 4 порядка. Была написана реализация данного метода на языке программирования Python.

Как выше было сказано, данный метод широко используют в решениях дифференциальных уравнений различных порядков. Чтобы повысить точность данного метода, можно:
\begin{enumerate}
\item Повысить порядок метода Рунге-Кутты
\item Увеличить количество шагов
\end{enumerate}
Но повышение порядка метода приведёт к вычислительной трудности, заключающийся в вычислении на каждом этапе N промежуточных значений. При этом изменение шага для методов Рунге-Кутты сложности не представляет. 

Метод Рунге-Кутты - довольно широко используемый метод для решения дифференциальных уравнений, так как при небольшом объеме вычислений он обладает высокой точностью метода, в отличие от стандартного метода Эйлера. Но при этом нельзя забывать, что существуют и другие типы методов (многошаговые методы), которые обладают еще более высокой точностью, но при этом более сложны в понимании и реализации. 

Поэтому самым оптимальным методом для решения задачи Коши дифференциальных уравнений любых порядков является классический метод Рунге-Кутты (4 порядка).


\end{document}
